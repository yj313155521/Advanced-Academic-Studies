% Options for packages loaded elsewhere
\PassOptionsToPackage{unicode}{hyperref}
\PassOptionsToPackage{hyphens}{url}
%
\documentclass[
]{article}
\usepackage{amsmath,amssymb}
\usepackage{lmodern}
\usepackage{iftex}
\ifPDFTeX
  \usepackage[T1]{fontenc}
  \usepackage[utf8]{inputenc}
  \usepackage{textcomp} % provide euro and other symbols
\else % if luatex or xetex
  \usepackage{unicode-math}
  \defaultfontfeatures{Scale=MatchLowercase}
  \defaultfontfeatures[\rmfamily]{Ligatures=TeX,Scale=1}
\fi
% Use upquote if available, for straight quotes in verbatim environments
\IfFileExists{upquote.sty}{\usepackage{upquote}}{}
\IfFileExists{microtype.sty}{% use microtype if available
  \usepackage[]{microtype}
  \UseMicrotypeSet[protrusion]{basicmath} % disable protrusion for tt fonts
}{}
\makeatletter
\@ifundefined{KOMAClassName}{% if non-KOMA class
  \IfFileExists{parskip.sty}{%
    \usepackage{parskip}
  }{% else
    \setlength{\parindent}{0pt}
    \setlength{\parskip}{6pt plus 2pt minus 1pt}}
}{% if KOMA class
  \KOMAoptions{parskip=half}}
\makeatother
\usepackage{xcolor}
\usepackage[margin=1in]{geometry}
\usepackage{graphicx}
\makeatletter
\def\maxwidth{\ifdim\Gin@nat@width>\linewidth\linewidth\else\Gin@nat@width\fi}
\def\maxheight{\ifdim\Gin@nat@height>\textheight\textheight\else\Gin@nat@height\fi}
\makeatother
% Scale images if necessary, so that they will not overflow the page
% margins by default, and it is still possible to overwrite the defaults
% using explicit options in \includegraphics[width, height, ...]{}
\setkeys{Gin}{width=\maxwidth,height=\maxheight,keepaspectratio}
% Set default figure placement to htbp
\makeatletter
\def\fps@figure{htbp}
\makeatother
\setlength{\emergencystretch}{3em} % prevent overfull lines
\providecommand{\tightlist}{%
  \setlength{\itemsep}{0pt}\setlength{\parskip}{0pt}}
\setcounter{secnumdepth}{-\maxdimen} % remove section numbering
\newlength{\cslhangindent}
\setlength{\cslhangindent}{1.5em}
\newlength{\csllabelwidth}
\setlength{\csllabelwidth}{3em}
\newlength{\cslentryspacingunit} % times entry-spacing
\setlength{\cslentryspacingunit}{\parskip}
\newenvironment{CSLReferences}[2] % #1 hanging-ident, #2 entry spacing
 {% don't indent paragraphs
  \setlength{\parindent}{0pt}
  % turn on hanging indent if param 1 is 1
  \ifodd #1
  \let\oldpar\par
  \def\par{\hangindent=\cslhangindent\oldpar}
  \fi
  % set entry spacing
  \setlength{\parskip}{#2\cslentryspacingunit}
 }%
 {}
\usepackage{calc}
\newcommand{\CSLBlock}[1]{#1\hfill\break}
\newcommand{\CSLLeftMargin}[1]{\parbox[t]{\csllabelwidth}{#1}}
\newcommand{\CSLRightInline}[1]{\parbox[t]{\linewidth - \csllabelwidth}{#1}\break}
\newcommand{\CSLIndent}[1]{\hspace{\cslhangindent}#1}
\ifLuaTeX
  \usepackage{selnolig}  % disable illegal ligatures
\fi
\IfFileExists{bookmark.sty}{\usepackage{bookmark}}{\usepackage{hyperref}}
\IfFileExists{xurl.sty}{\usepackage{xurl}}{} % add URL line breaks if available
\urlstyle{same} % disable monospaced font for URLs
\hypersetup{
  pdftitle={An Analysis of Data Ethics in the Internet Environment},
  pdfauthor={Jin Yan},
  hidelinks,
  pdfcreator={LaTeX via pandoc}}

\title{An Analysis of Data Ethics in the Internet Environment}
\author{Jin Yan}
\date{2022-09-21}

\begin{document}
\maketitle

\hypertarget{abstract}{%
\section{Abstract}\label{abstract}}

This paper first discusses the value of the large amount of data
available in the Internet environment, mainly related to medical health,
advertising and credit building. It then demonstrates the problems of
data misuse in the current Internet context. The importance of
establishing data ethics is illustrated by comparison. At the end of the
paper, the current state of personal data protection in Chinese society
is presented, along with a specific analysis of the current situation.

\hypertarget{introduction}{%
\section{Introduction}\label{introduction}}

The development of machine learning and artificial intelligence has made
the footprints people leave on the Internet particularly valuable.
Currently, the records of people's activities on the Internet, whether
in the medical, advertising or financial sectors, can be used by the
holders of these data to create enormous wealth. At the same time,
ordinary Internet users are increasingly aware of the importance of
their personal data. In the process of using these data, there are also
things that can harm people's interests. There are heated debates on how
to protect their data. The main purpose of this paper is to raise
awareness of data ethics by showing the current problems of data misuse
in the context of the Internet and the responses of various countries.

\hypertarget{literature-review}{%
\section{Literature review}\label{literature-review}}

Personal data has contributed to the development of the following
industries.

\hypertarget{medical-industry}{%
\subsection{Medical Industry}\label{medical-industry}}

Machine learning and large-scale data sets have allowed the full
potential of personal information to be exploited.{[}1{]}Specifically,
there are now many fitness apps and diet intake apps that allow users to
extract health information from recorded data, and the value of this
data rises as the analytical and predictive capabilities of these apps
improve. In addition, large-scale medical information can be used to
treat diseases that could not be dealt with before. {[}1{]} Even some
recent analyses suggest that the value of data needs to be reevaluated
because the current ability to analyze and compile data has changed
dramatically from what it was before. {[}2{]} In fact, the market for
medical data in the United States is in the billions. And there are
already startups in the U.S. that are profiting from providing
individuals with secure access to selling their personal information.

\hypertarget{advertising-industry}{%
\subsection{Advertising Industry}\label{advertising-industry}}

In the ``Big Data Era'', data has become an effective input to the wide
range of services offered by the Internet. With this data, Internet
platform companies can provide customized advertising recommendations.
The data shows that the conversion rate of advertisements has become
twice as high as before compared to ordinary media.{[}3{]}

\hypertarget{financial-sector}{%
\subsection{Financial sector}\label{financial-sector}}

The most far-reaching change that technological advances have brought
about for China is the creation of a social credit system. This system
will evaluate and grade the creditworthiness of 1.4 billion people, by
analyzing their social behavior and collecting economic and government
data.{[}4{]} For example, Tencent Credit, a subsidiary of Tencent
Technology, has been collecting large amounts of data through QQ and
other security applications {[}5{]} and in the process has found that
data generated by social media can effectively help in credit risk
analysis.{[}6{]}

\hypertarget{problems-in-data-use-and-some-solutions.}{%
\subsection{Problems in data use and some
solutions.}\label{problems-in-data-use-and-some-solutions.}}

Although the great value of medical data is valued by companies and
research institutions, there are a large number of people who are not
well aware of the value of their data.{[}7{]} In response to this
situation those medical researchers who collect and use medical data,
government agencies and companies should make vulnerable groups aware of
the potential value of their own data.{[}1{]}Governments have also taken
some measures to protect personal data. For example, in Europe the
latest General Data Protection Regulation has recognized personal data
as the property of individuals.

\hypertarget{personal-data-specifics-and-analysis-in-china}{%
\section{Personal data specifics and analysis in
China}\label{personal-data-specifics-and-analysis-in-china}}

\hypertarget{specifics}{%
\subsection{specifics}\label{specifics}}

The development of big data and artificial intelligence technologies in
China has led to a significant rise in the digital economy. It was able
to reach 26 trillion RMB in 2017. The contribution to China's GDP
reached 32\%. At the same time, the growth rate of the digital economy
was 18\%, while the overall economic growth rate was 6.9\%.{[}8{]} It is
also important to note that the use of data in China has created some
problems. Approximately 70\% of the population's personal information is
compromised on the Internet. This information includes names, addresses,
phone numbets preparing for exams also receive phone calls and messages
from people selling them preparation materials that they may use for the
exam.{[}9{]} There are technology companies that help government public
safety agencies develop surveillance systems through the use of big
data, voice recognition, and image recognition.{[}4{]} Data breaches can
even lead to serious cases, and the Xu Yu Yu case is very typical. A
high school graduate girl lost her college tuition because she was
cheated with 8000 RMB, and then became depressed and had an acute heart
attack. The reason was that the fraudster got her information by
attacking the school database and thus won the girl's trust by using the
real information in the database.{[}10{]}

In fact, China has enacted numerous laws and regulations to address
these issues: in November 2016, the ``National People's Congress''
enacted the Internet Security Law.{[}11{]} China's criminal law has also
had two amendments in the last decade to address personal information
security.{[}11{]}

\hypertarget{analysis}{%
\subsection{Analysis}\label{analysis}}

Although there is already a considerable amount of legislation in place
to safeguard personal data, the current enforcement of the law is
somewhat unsatisfactory. Despite the high number of cases related to
personal information from 2010-2015, the number of actual convictions is
relatively low.In addition, although the amendment to the criminal code
increased the length of imprisonment to three to seven years, the most
serious convictions between 2015 and 2016 were for two years, with two
years of probation.In fact, the main reason for these problems is that
the public is not well aware of the potential risks associated with
personal data breaches. The public is also not very sensitive to the
security of personal data. In such a situation, personal data can be
easily obtained. This also affects the enforcement of legislation and
laws. In such a situation, it is particularly important to build a
consensus on data ethics.

\hypertarget{conclusion}{%
\section{Conclusion}\label{conclusion}}

A questionnaire was used to investigate the perception of data
protection among the 20-30 year old Chinese adult population. 100\% of
the respondents experienced problems due to information leakage. What is
worth exploring is that when asked how much money they would be willing
to spend to protect their data, 80\% chose not to spend in this area.
This result shows that the problem of personal data leakage is indeed
common, but for most people the impact of online information leakage on
their lives is relatively limited. Limited by the sample size, the
samples involved in this survey study are all bachelor's degree holders
with certain risk-averse ability. When asked whether they care about
Internet companies recording their own behavioral data on the Internet
for targeted advertising, 100\% of people feel uncomfortable, and 40\%
of them feel very much so. Combined with the previous information, it
can also be concluded that this group has a relatively strong sense of
personal data protection. However, the awareness of the risks caused by
data leakage is not quite sufficient.

\hypertarget{references}{%
\section*{References}\label{references}}
\addcontentsline{toc}{section}{References}

\hypertarget{refs}{}
\begin{CSLReferences}{0}{0}
\leavevmode\vadjust pre{\hypertarget{ref-dulhanty_present_2021}{}}%
\CSLLeftMargin{{[}1{]} }%
\CSLRightInline{Dulhanty A 2021
\href{https://doi.org/10.2471/BLT.19.237248}{Present value of future
health data: Ethics of data collection and use} \emph{Bull World Health
Organ} \textbf{99} 162--3}

\leavevmode\vadjust pre{\hypertarget{ref-foster_embedded_2018}{}}%
\CSLLeftMargin{{[}2{]} }%
\CSLRightInline{Foster J and Clough P 2018
\href{https://doi.org/10.1002/asi.23987}{Embedded, added, cocreated:
{Revisiting} the value of information in an age of data} \emph{Journal
of the Association for Information Science and Technology} \textbf{69}
744--8}

\leavevmode\vadjust pre{\hypertarget{ref-beales_value_2022}{}}%
\CSLLeftMargin{{[}3{]} }%
\CSLRightInline{Beales H 2022 The {Value} of {Behavioral} {Targeting}}

\leavevmode\vadjust pre{\hypertarget{ref-__2017}{}}%
\CSLLeftMargin{{[}4{]} }%
\CSLRightInline{孟建国K B 2017
\href{https://cn.nytimes.com/business/20171205/china-artificial-intelligence/}{人工智能在中国:技术背后的反乌托邦可能性}
\emph{纽约时报中文网}}

\leavevmode\vadjust pre{\hypertarget{ref-noauthor__nodate}{}}%
\CSLLeftMargin{{[}5{]} }%
\CSLRightInline{Anon
\href{http://finance.sina.com.cn/roll/20150825/080823054761.shtml}{芝麻信用押注消费、腾讯征信紧贴机构
互联网征信各展所长\_滚动新闻\_新浪财经\_新浪网}}

\leavevmode\vadjust pre{\hypertarget{ref-noauthor__nodate-1}{}}%
\CSLLeftMargin{{[}6{]} }%
\CSLRightInline{Anon
\href{https://finance.caixin.com/2015-07-21/100830962.html}{腾讯征信:社交数据也能评估信用
\_金融频道\_财新网}}

\leavevmode\vadjust pre{\hypertarget{ref-cheah_challenges_2018}{}}%
\CSLLeftMargin{{[}7{]} }%
\CSLRightInline{Cheah P Y, Jatupornpimol N, Hanboonkunupakarn B,
Khirikoekkong N, Jittamala P, Pukrittayakamee S, Day N P J, Parker M and
Bull S 2018 \href{https://doi.org/10.1186/s12910-018-0326-x}{Challenges
arising when seeking broad consent for health research data sharing: A
qualitative study of perspectives in {Thailand}} \emph{BMC Medical
Ethics} \textbf{19} 86}

\leavevmode\vadjust pre{\hypertarget{ref-weber_comparative_2020}{}}%
\CSLLeftMargin{{[}8{]} }%
\CSLRightInline{Weber P A, Zhang N and Wu H 2020
\href{https://doi.org/10.1007/s10660-020-09422-3}{A comparative analysis
of personal data protection regulations between the {EU} and {China}}
\emph{Electron Commer Res} \textbf{20} 565--87}

\leavevmode\vadjust pre{\hypertarget{ref-noauthor_130_nodate}{}}%
\CSLLeftMargin{{[}9{]} }%
\CSLRightInline{Anon
\href{http://news.sina.com.cn/c/2014-12-10/032331266499.shtml}{130万考研者信息被叫卖
专家吁出台信息保护法{\textbar}个人信息\_新浪新闻}}

\leavevmode\vadjust pre{\hypertarget{ref-feng_future_2019}{}}%
\CSLLeftMargin{{[}10{]} }%
\CSLRightInline{Feng Y 2019
\href{https://doi.org/10.1080/10192557.2019.1646015}{The future of
{China}'s personal data protection law: Challenges and prospects}
\emph{Asia Pacific Law Review} \textbf{27} 62--82}

\leavevmode\vadjust pre{\hypertarget{ref-han_market_2017}{}}%
\CSLLeftMargin{{[}11{]} }%
\CSLRightInline{Han D 2017
\href{https://doi.org/10.17645/mac.v5i2.890}{The market value of who we
are: The flow of personal data and its regulation in {China}}
\emph{Media and Communication} \textbf{5} 21--30}

\end{CSLReferences}

\end{document}
